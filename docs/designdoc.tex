%% LyX 2.3.7 created this file.  For more info, see http://www.lyx.org/.
%% Do not edit unless you really know what you are doing.
\documentclass[12pt,english]{article}
\usepackage[T1]{fontenc}
\usepackage[latin9]{inputenc}
\usepackage{float}

\makeatletter

%%%%%%%%%%%%%%%%%%%%%%%%%%%%%% LyX specific LaTeX commands.
%% Because html converters don't know tabularnewline
\providecommand{\tabularnewline}{\\}

%%%%%%%%%%%%%%%%%%%%%%%%%%%%%% User specified LaTeX commands.
\usepackage{tikz}
\usepackage{circuitikz}
\usetikzlibrary{shapes, arrows}

\makeatother

\usepackage{babel}
\begin{document}
\title{\textbf{2024 Design Document}\\
\emph{Secure MISC}}
\author{0xDACC}
\maketitle
\begin{center}
\newpage{}
\par\end{center}

\section{Proposed Attest Changes}

\paragraph{Store attestation PIN as a hash with enough rounds that it takes
approximately 2 seconds.}
\begin{itemize}
\item Limits brute force attempts
\item Makes raw PIN unable to be extracted from flash
\end{itemize}

\paragraph{Store attestation data encrypted with symmetric key as 1 round less
of attestation pin hash}
\begin{itemize}
\item Also limits brute force and makes PIN unreadable from flash
\end{itemize}

\paragraph{Meets SR3 and SR4}

\noindent \newpage{}

\section{Proposed Replace Changes}

\paragraph{Store replacement token as a hash}
\begin{itemize}
\item Makes token unable to be extracted from flash
\item Highly unlikely that the token can be brute forced
\end{itemize}

\paragraph{Verify component authenticity}
\begin{enumerate}
\item Store an asymmetric public key in flash
\item Generate a random number using onboard TRNG
\item Ask new component to sign random number
\item Verify using onboard public key
\end{enumerate}

\paragraph{Meets SR3}

\newpage{}

\section{Proposed Boot Changes}

\paragraph{Verify integrity of all 3 boards}
\begin{itemize}
\item Store public keys A and D and private keys B and C on AP
\item Store public key B and private key A on Component1
\item Store public key C and private key D on Component2
\end{itemize}
\noindent \begin{center}
\noindent \begin{center}
\begin{figure}[!ht]
\centering
\resizebox{1\textwidth}{!}{%
\begin{circuitikz}
\tikzstyle{every node}=[font=\LARGE]
\draw [rounded corners = 3.0] (25,24.5) rectangle  node {\LARGE C2} (22.5,25.75);
\draw [rounded corners = 3.0] (18.75,25.75) rectangle  node {\LARGE AP} (21.25,24.5);
\draw [rounded corners = 3.0] (15,25.75) rectangle  node {\LARGE C1} (17.5,24.5);
\draw [<-, >=Stealth, dashed] (17.5,25.5) -- (18.75,25.5);
\draw [<-, >=Stealth, dashed] (18.75,24.75) -- (17.5,24.75);
\draw [<-, >=Stealth, dashed] (21.25,25.5) .. controls (22,25.5) and (22,25.5) .. (22.5,25.5);
\draw [<-, >=Stealth, dashed] (22.5,24.75) -- (21.25,24.75);
\node [font=\LARGE] at (18.25,26.25) {1};
\node [font=\LARGE] at (18.25,24) {2};
\node [font=\LARGE] at (21.75,26.25) {3};
\node [font=\LARGE] at (21.75,24) {4};
\end{circuitikz}
}%
\label{fig:my_label} \end{figure}
\par\end{center}
\par\end{center}
\begin{enumerate}
\item AP verifies Component1
\begin{enumerate}
\item AP generates a random number and asks Component1 to sign with key
A
\item AP verifies signature using key A
\end{enumerate}
\item Component1 verifies AP
\begin{enumerate}
\item Component1 generates a random number and asks AP to sign with key
B
\item Component1 verifies signature using key B
\end{enumerate}
\item Component2 verifies AP
\begin{enumerate}
\item Component2 generates a random number and asks AP to sign with key
C
\item Component2 verifies signature using key C
\end{enumerate}
\item AP verifies Component2
\begin{enumerate}
\item AP generates a random number and asks Component2 to sign with key
D
\item AP verifies signature using key D
\end{enumerate}
\end{enumerate}

\paragraph{\emph{If any signatures are invalid, stop immediately and shut down.}}

\begin{figure}[!ht]
\centering
\resizebox{1\textwidth}{!}{%
\begin{circuitikz}
\tikzstyle{every node}=[font=\LARGE]
\draw [rounded corners = 3.0] (12.5,12) rectangle  node {\large Boot Flow} (15,10.75);
\draw [rounded corners = 3.0] (12.5,9.5) rectangle  node {\tiny All systems valid?} (15,8.25);
\draw [](13.75,10.75) to[short] (13.75,9.5);
\draw [ color={rgb,255:red,224; green,27; blue,36}, short] (12.5,9) -- (10,9);
\draw [ color={rgb,255:red,51; green,209; blue,122}, short] (15,9) -- (17.5,9);
\draw [ color={rgb,255:red,51; green,209; blue,122}, short] (17.5,9) -- (17.5,7);
\draw [ color={rgb,255:red,224; green,27; blue,36}, short] (10,9) -- (10,7);
\node [font=\tiny] at (11.25,8.5) {Send Fail Packet};
\draw [ fill={rgb,255:red,224; green,27; blue,36} ] (10,6.25) circle (0.75cm) node {\Large Die} ;
\node [font=\tiny] at (16.25,8.5) {Send Boot to C1};
\draw [ rounded corners = 3.0 ] (16.25,7) rectangle  node {\tiny Valid ACK Received?} (18.75,5.75);
\draw [ color={rgb,255:red,224; green,27; blue,36}, ] (16.25,6.5) to[short] (10.5,6.5);
\draw [ color={rgb,255:red,51; green,209; blue,122}, ](18.75,6.5) to[short] (21.25,6.5);
\draw [ color={rgb,255:red,51; green,209; blue,122}, ](21.25,6.5) to[short] (21.25,4.5);
\node [font=\tiny] at (20,6) {Send Boot to C2};
\draw [ rounded corners = 3.0 ] (20,4.5) rectangle  node {\tiny Valid ACK Received?} (22.5,3.25);
\draw [ color={rgb,255:red,224; green,27; blue,36}, short] (20,4) -- (10,4);
\draw [ color={rgb,255:red,224; green,27; blue,36}, short] (10,4) -- (10,5.5);
\draw [ color={rgb,255:red,51; green,209; blue,122}, short] (22.5,4) -- (25,4);
\node [font=\tiny] at (23.75,3.5) {Boot AP};
\draw [ color={rgb,255:red,51; green,209; blue,122}, short] (25,4) -- (25,2);
\draw [ fill={rgb,255:red,51; green,209; blue,122} ] (25,2) circle (0.75cm) node {\Large Boot} ;
\end{circuitikz}
}%

\label{fig:my_label}
\end{figure}

\noindent 
\begin{table}[H]
\noindent \centering{}%
\begin{tabular}{|c|c|c|c|c|}
\hline 
\multicolumn{2}{|c|}{Header} & \multicolumn{3}{c|}{Payload}\tabularnewline
\hline 
Packet Magic & Checksum & Length & Data & Signature\tabularnewline
\hline 
(1 byte) & (3 bytes) & (1 byte) & (4 bytes) & (65 bytes)\tabularnewline
\hline 
0xBB &  & 0x45 & 0x424F4F54 & \tabularnewline
\hline 
\end{tabular}\caption{Component Boot Packet}
\end{table}

\noindent 
\begin{table}[H]
\centering{}%
\begin{tabular}{|c|c|c|c|c|}
\hline 
\multicolumn{2}{|c|}{Header} & \multicolumn{3}{c|}{Payload}\tabularnewline
\hline 
Packet Magic & Checksum & Length & Data & Signature\tabularnewline
\hline 
(1 byte) & (3 bytes) & (1 byte) & (1 byte) & (65 bytes)\tabularnewline
\hline 
0xAA &  & 0x42 & 0xFF or 0x00 & \tabularnewline
\hline 
\end{tabular}\caption{Startup ACK Packet}
\end{table}


\paragraph{If:}
\begin{itemize}
\item Packet Magic != 0xBB or 0xAA
\item CSUM(Payload) != Expected Checksum
\item Length != 0x45
\item Data != 0x424F4F54
\item recover(signature) != key B or key C
\item Startup ACK data == 0xFF and recover(signature) == key A, key B, key
C, or key D
\end{itemize}
\textbf{\emph{Shut down immediately, send fail packet if running on
component, and do not continue operation.}}

\paragraph{Meets SR1 and SR2}

\newpage{}

\section{Proposed Secure TX Changes}

\paragraph{ECIES Based Scheme}
\begin{itemize}
\item Generate private key using RNG
\item Create an encrypted channel even though unnecessary.
\item Confidentiality will be provided to make RE'ing just a tiny bit harder
\item Encrypt packets with negotiated key
\item Negotiate HMAC key over new channel
\item Append HMAC to all packets before encrypting
\item Calculate checksum of encrypted data
\end{itemize}
\noindent 
\begin{table}[H]
\centering{}%
\begin{tabular}{|c|c|c|c|c|c|c|}
\hline 
\multicolumn{2}{|c|}{Header} & \multicolumn{5}{c|}{Encrypted Payload}\tabularnewline
\hline 
Packet Magic & Checksum & Payload Magic & Length & Nonce & Data & HMAC\tabularnewline
\hline 
(1 byte) & (3 bytes) & (1 byte) & (1 byte) & (6 bytes) & (255 bytes?) & (32 bytes)\tabularnewline
\hline 
0xEE &  & 0xDD &  &  &  & \tabularnewline
\hline 
\end{tabular}\caption{Encrypted I2C Packet}
\end{table}
\begin{table}[H]
\centering{}%
\begin{tabular}{|c|c|c|c|c|}
\hline 
\multicolumn{5}{|c|}{Key Exchange}\tabularnewline
\hline 
Packet Magic & Checksum & Length & Key Material & Key Hash\tabularnewline
\hline 
(1 byte) & (3 bytes) & (1 byte) & (32 bytes) & (32 bytes)\tabularnewline
\hline 
0x4A or 0x4B &  & 0x40 &  & \tabularnewline
\hline 
\end{tabular}\caption{Key Exchange I2C Packet}
\end{table}


\paragraph{If:}
\begin{itemize}
\item Packet Magic != Expected Magic
\item CSUM(packet) != Expected Checksum
\item Payload Magic != Expected Magic
\item HMAC(Data) != HMAC or Hash(Key) != Key Hash
\item Nonce != Expected Nonce
\end{itemize}
\textbf{\emph{Shut down immediately, send fail packet, and do not
continue operation.}}

\paragraph{Meets SR5}

\newpage{}

\section{Other}

\paragraph{Secure DAPLink firmware for RISC-V chip}
\begin{itemize}
\item Only execute signed code
\item Disable the DAPLINK flashing utility
\item Disable code debugging
\item Disable MAINTENANCE mode
\end{itemize}

\paragraph{Secure key storage}
\begin{itemize}
\item All asymmetric and symmetric keys located on flash will be stored
in an encrypted state
\item Wrapper keys will be compile-time constants and XOR'ed with another
compile-time constant so the raw key will \emph{NEVER} be stored in
flash
\item By wrapping all keys, a flash dumper payload would not be able to
extract the real keys and static reverse engineering would have a
similar outcome
\end{itemize}

\paragraph{All of the above objectives are futile if the attacker can simply
modify the flash or just set a breakpoint where the validation happens.
By not allowing the chip to be debugged (easily) and only allowing
signed code to be run, security becomes a lot more reasonable. After
reading through the requirements, some of these ``secure boot''
steps may be unnecessary, so may or may not be implemented.}

\newpage{}

\section{Summary}

\subsection{SR1 All components must be valid for AP to boot}
\begin{itemize}
\item Validate Component1 integrity through signing an arbitrary number
\item Validate Component2 integrity in same manner
\item Components then validate the AP to make sure all 3 systems are present
and valid
\item Boot the AP
\end{itemize}

\subsection{SR2 All components must be validated by AP and commanded before booting}
\begin{itemize}
\item After a successful handshake, it can be assumed that all components
are valid
\item Send signed boot command to components from AP
\item Boot individual components
\end{itemize}

\subsection{SR3 The Attestation PIN and Replacement Token should be kept confidential}
\begin{itemize}
\item PIN will be stored as a hash with enough iterations to reduce the
brute force likelihood
\item Replacement Token will also be stored as a hash
\end{itemize}

\subsection{SR4 Component Attestation Data should be kept confidential}
\begin{itemize}
\item Attestation Data will be stored with symmteric encryption with the
key being derived from the Attestation PIN
\end{itemize}

\subsection{SR5 Integrity and Authentication of all communications}
\begin{itemize}
\item All messages will follow a standard packet format with a negotiated
HMAC key and assymetric encryption
\item A nonce and ephermeral keys may be included to limit replay attacks
\end{itemize}

\end{document}
